\documentclass{article}
\usepackage{amsmath}
\usepackage{listings}
\usepackage{xcolor}
\lstset{
  language=bash,
  basicstyle=\ttfamily,
  columns=fullflexible,
  breaklines=true,
  %postbreak=\mbox{\textcolor{red}{$\hookrightarrow$}\space},
}

\begin{document}
\title{Makefiles and You}
\author{James Biddle}
\maketitle
\section{Part A}
\begin{enumerate}
\item Create a file called 'makefile', 'Makefile' or 'GNUmakefile'
\item Make a target called 'target'
\item Make it echo some text when run, highlight the fact that all you're doing with the makefile is executing bash commands.
\item Note that to write a command it must be preceeded by a TAB
\item Now make a new target called 'program.x', to compile 'program.f90'. Do not include dependency yet
\item Edit 'program.f90' to write another line of text
\item Type make and show that it doesn't recompile because there is no dependency specified.
\item Add dependency to 'program.f90'
\item Look at what happens if we just type make
\item Add dummy 'all' target to run all targets
\item Make 'clean' target to remove generated files
\item Make a file called 'clean' and then type 'make clean'. Show that we get undesired behaviour
\item Introduce a '.PHONY' target to resolve this behaviour
\item We've written 'program' a lot, which introduces concerns for typos. We can resolve this with inbuilt variables:
\pagebreak
\begin{lstlisting}
$@ to represent the full target name of the current target
$? returns the dependencies that are newer than the current target
$* returns the text that corresponds to % in the target
$< returns the name of the first dependency
$^ returns the names of all the dependencies with space as the delimiter
\end{lstlisting}
 
\end{enumerate}
\end{document}